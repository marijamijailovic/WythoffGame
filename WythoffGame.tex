% !TEX encoding = UTF-8 Unicode

\documentclass[a4paper]{article}

\usepackage{xcolor}
\usepackage{url}
\usepackage[T2A]{fontenc} % enable Cyrillic fonts
\usepackage[utf8]{inputenc} % make weird characters work
\usepackage[margin=1in]{geometry}
\usepackage[serbianc]{babel}
\usepackage{CJKutf8}
\usepackage{graphicx}
\usepackage{fancyhdr}
\usepackage{listings}
\usepackage{csvsimple}
\usepackage{pgfplotstable}
\usepackage{longtable}
\usepackage{float}
% set the default code style

\definecolor{ghostwhite}{rgb}{0.98, 0.98, 0.98}
\definecolor{mediumviolet-red}{rgb}{0.78, 0.08, 0.52}
\definecolor{hooker\'sgreen}{rgb}{0.0, 0.44, 0.0}
\lstset{
	language=C++,
	breaklines=true,
	backgroundcolor=\color{ghostwhite},
	frame=tb, % draw a frame at the top and bottom of the code block
	tabsize=2, % tab space width
	showstringspaces=false, % don't mark spaces in strings
	numbers=left, % display line numbers on the left
	numberstyle=\color{gray},
	rulecolor=\color{black},
	keywordstyle=\color{mediumviolet-red}, % keyword color
	commentstyle=\color{hooker\'sgreen} %comment color
}
%\usepackage[english,serbianc]{babel} %ukljuciti babel sa ovim opcijama, umesto gornjim, ukoliko se koristi cirilica

\usepackage[unicode]{hyperref}
\hypersetup{colorlinks,citecolor=green,filecolor=green,linkcolor=blue,urlcolor=blue}

\pagestyle{fancy}
\fancyhf{}
\renewcommand{\headrulewidth}{0pt}
\fancyfoot[R]{\thepage}

\graphicspath{{./src/statistics/picture/}}

\begin{document}
\begin{titlepage}
    \begin{center}
        \vspace{0.5cm}
        
        \Large{
	        Универзитет у Београду\\
	        Математички факултет\\
        }
    
        \vspace{0.5cm}
        \Large{Мастер рад}    
        
        \vspace{2.0cm}
        
        \Huge
        \rule[0.5cm]{\textwidth}{0.5pt}
        \textbf{Игра ним}
        \rule{\textwidth}{0.5pt}
        \vspace{0.5cm}
        
        \vspace{2.0cm}
        
        \begin{minipage}[t]{0.47\textwidth}
        	\textnormal{\large{\bf Аутор:\\}}
        	{\large Марија Мијаиловић}
        \end{minipage}\hfill\begin{minipage}[t]{0.47\textwidth}\raggedleft
        	\textnormal{\large{\bf Ментор:\\}}
        	{\large Др Миодраг Живковић}
        \end{minipage}
        
        \vfill
        
        {\Large Катедра за рачунарство и информатику}
        
        \vspace{0.8cm}
        
        \includegraphics[width=0.3\textwidth]{matf_logo.png}
        
        \large{Београд, мај 2020}
        
    \end{center}
\end{titlepage}
\newpage
\pagenumbering{roman}
\tableofcontents

\newpage
\pagenumbering{arabic}
\section{Увод}
\label{sec:uvod}



\section{Витхоф-ова игра}
\label{sec:vithofova_igra}

Витхоф-ова игра (енг.{~\em Wythoff's game})\cite{10.2307/2321643} је математичка стратешка игра за два играча. На талону су нам дате две гомиле жетона, играчи наизменично узимају жетоне са једне или обе гомиле. Приликом узимања жетона са обе гомилe, рецимо $ k (> 0) $ са једне и $ l (> 0) $ са друге број узетих жетона мора задовољити услов $ |k - l| < a $, где је $ a $ било који позитиван број. Игра се завршава када број жетона на талону буде нула, а онај играч који је уклонио последњи жетон или жетоне је победник. Прослеђивање није могуће - сваки играч када је на потезу мора да уклони бар један жетон.

У класичној Витхоф игри $ a = 1 $, што значи да ако играч узима жетоне са обе гомиле, број узетих жетона мора бити једнак.

Еквивалентни опис игре би био: Имамо једну шаховску краљицу постављну било где на табли, сваки играч може да помера краљицу произвољан број корака у правцу југа, запада, или југозапада. Победник је играч који први помери краљицу у доњи леви ћошак табле.\cite{cut-the-knot} \cite{singingbanana-youtube}

Постоје тврдње да се ова игра играла у Кини  под именом "\begin{CJK}{UTF8}{gbsn}捡石子\end{CJK} jiǎn shízǐ"(енг.{~\em picking stones}). \cite{Yaglom}

Холандски математичар В. А. Витхоф (енг.{~\em W. A. Wythoff}) је 1907. године објавио математичку анализу ове игре. \cite{wythoff1907modification}

\section{Оптимална стратегија}
\label{sec:optimalna_strategija}

Било која позиција се може представити паром бројева $ (a, b) $, где је $ a \le  b $, док  $ a $ и $ b $ представљају број жетона на талону или координате позиције краљице. Имамо два типа позиција око којих се врти игра, П-позиције и Н-позиције. На П-позицји, играч који је на потезу ће изгубити и са најбоље одиграним потезом, тачније претходни играч може да победи шта год одиграо противник. Док на Н-позицији, следећи играч може да победи шта год противник одиграо. %играч који је на потезу ће победити и са најбоље одиграним потезом.

Класификација позиција на П и Н се дефинише рекурзивно на следећи начин:
\begin{enumerate}
	\item $ (0, 0) $ је П-позиција јер играч који је на потезу не може да одигра ниједан валидан потез, па је његов противник победник.
	\item Било која позиција са које је П-позиција достижна је Н-позиција. 
	\item Ако сваки потез води ка Н-позицији, онда је то П-позиција.
\end{enumerate}

На пример, све позиције облика $ (0, b) $ и $ (b, b) $, где је $ b > 0 $ су Н-позиције, на основу другог правила. За $ a = 1 $ позиција  $ (1, 2) $  је П-позиција, зато што су са ње достижне само позиције $ (0, 1), (0, 2), (1, 0) $ и $ (1, 1) $, које су Н-позиције. Још неке П-позиција су $ (0, 0), (1, 2), (3, 5), (4, 7), (6, 10) $ и $ (8, 13) $. 

Да би се Витхоф игра ирала на најбољи могући начин, потребно је знати две ствари:
\begin{itemize}
	\item Препознати припроду тренутне позиције, да ли је П или Н
	\item Израчунати следећи потез, уколико је тренутна позиција Н
\end{itemize}

Разлог битности лежи у чињеници да уколико је тренутна позиција Н, знамо да постоји потез који нас води на П-позицију, а тај потез можемо израчунати и победити. Са друге ако је тренутна позиција П не можемо урадити ништа, само одиграти произвољан валидан потез и надати се најбољем, с обзиром на то да се у једном потезу са П-позиције стиже на Н-позицију, са које противник може да победи ако зна да израчуна П-позицију. У овом раду биће приказано како се може израчунати победничка позиција, користећи рекурзивну, алгебарску или аритметичку стратегију.

\section{Рекурзивна стратегија}
\label{sec:rekurzivna_strategija}

Рекурзивном стратегијом П-позиције добијају се рачунајући $ B_{n} - A_{n} = an $. За $ A_n $ важи $ A_{n} = mex \{ A_{i}, B_{i} : i < n \} $, где $ mex $ дефинишемо као најмању вредност целог сортираног скупа, који не припада подскупу, тачније то је најмања вредност комплементарног скупа. Треба напоменути да је $ mex \emptyset = 0 $.

У случају да се играч помера са $ (A_{n}, B_{n}) $ позиције, и узима само жетоне са једне гомиле, тим потезом производи позицију која није облика $ (A_{i}, B_{i}) $. Уколико узима жетоне са обе гомиле такође производи потез који није облика $ (A_{i}, B_{i}) $, у супротном уколико би произведена позиција била $ (A_{i}, B_{i}) $, морало би да важи $ |(B_{n} - B_{i}) - (A_{n}-A_{i})| < a $, ако искористимо да је $ B_{n} - A_{n} = an $ добијамо да треба да буде задовољено $ |(n-i)a| < a $, што је тачно само ако је $ i = n $, што је контрадикција. 

У случају да се играч помера са позиције $ (x, y), x \le y $, позиција која није облика $ (A_{i}, B_{i}), i \ge 0 $. Како су $ A $ и $ B $ комплементарни скупови, може се сматрати да је $ x = B_{n} $, или је $ x = A_{n} $, за $ n \ge 0 $ .
\begin{itemize}
	\item Случај 1: $ x = B_{n} $ онда $ y = A_{n} $
	\item Случај 2: $ x = A_{n} $, ако је $ y > B_{n} $ онда $ y = B_{n} $. Док у случају када је $ A_{n} \le y < B_{n} $ онда рачунамо $ d = y - x, m = \lfloor \frac{d}{a} \rfloor $ и померамо се на позицију $ (A_{m}, B_{m}) $. Ово је легалан потез јер:
		\begin{enumerate}
			\item $ d = y - A_{n} < B_{n} - A_{n} = an $, стога $ m = \lfloor \frac{d}{a} \rfloor \le \frac{d}{a} < n $
			\item $ y = A_{n} + d \ge A_{m} + am = B_m $
			\item $ |(y - B_{m}) - (x - A_{m})| = |d - am| < a $
		\end{enumerate}
\end{itemize}

\section{Алгебарска стратегија}
\label{sec:algebarska_strategija}

Алгебарском стратегијом П-позиције добијају се рачунајући $ A'_{n} = \lfloor n\alpha \rfloor $, и $ B'_{n} = \lfloor n\beta \rfloor $, где $ \alpha $ и $ \beta $ рачунамо:  
\begin{center}
	$ \alpha = \frac{2 - a + \sqrt{a^2 + 4}}{2} $ , $ \beta = \alpha + a $ 
\end{center}
Где је $ \alpha $ позитиван корен квадратне једначине $ \xi^{-1} + (\xi+a)^{-1} = 1 $, тако су $ \alpha $ и $ \beta $ ирационални за сваки позитиван број $ a $, и задовољавају $ \alpha^{-1} + \beta^{-1} = 1 $

Уочимо да је $ A'_{0} = 0, B'_{0} = 0 $ и $ B'_{n} - A'_{n} = an $. Такође како су $ A'_{n} $ и $ B'_{n} $ растући низови и комплементарни, то важи још и да је $ A'_{n} = mex \{ A'_{i}, B'_{i} : i < n \} $. Што показује да је $ A'_{n} = A_{n} $ и $ B'_{n} = B_{n}  $ за $ n \ge 0 $.

Тако да се надаље за игру може спроводити иста стратегија описана у \ref{sec:rekurzivna_strategija}. 

\section{Аритметичка стратегија}
\label{sec:aritmeticka_strategija}

Дефинишемо $ p $ и $ q $ низове рекурзивно на следећи начин :

\begin{center}
	$ p_{-1} = 1, p_{0} = а_{0}, p_{n} = a_{n}p_{n-1} + p_{n-2}, (n \geq 1 ) $\\
	$ q_{-1} = 0, q_{0} = 1, q_{n} = a_{n}q_{n-1} + q_{n-2}, (n \geq 1 ) $
\end{center}

Где је $ a_{0}, a_{1}, ... $ јединствени бесконачни низ природних бројева за које важи $ a_{0} = 1 $ и $ a_{1}, а_{2}, ... ,  $ су позитивни и $ a_{n} \ne 1 $, тако да уколико је $ \alpha $ ирационалан број можемо га представити следећим једноставним бесконачним верижним разломком:

\begin{center}
		$ \alpha = 1 + \frac{1}{a_{1} + \frac{1}{a_{2} + \frac{1}{a_{3} + ...}}} = [1, a_{1}, , a_{2}, , a_{3}, ...] $ 
\end{center}

Из ове једнакости се може закључити да је 
\begin{center}
	$ \frac{p_{n}}{q_{n}} = [1, a_{1}, , a_{2}, , a_{3}, ..., a_{n}] $.
\end{center}

Тачније $ \frac{p_{n}}{q_{n}} $ је конвергент ирационалног броја $ \alpha $. 

Сваки рационални број $ \frac{m}{n} $се Еуклидовим алгоритмом може претворити у коначни једноставни верижни разломак.

\begin{center}
	$ m = nq + r \Rightarrow
	  \frac{m}{n} = q + \frac{r}{n} = q + \frac{1}{\frac{n}{r}} $
\end{center}
Процес се даље наставља дељењем $ n $ са $ r $.

Потребно је увести $ p $-систем и $ q $-системе нумерације. 
У $ p $-систему можемо записати сваки позитиван број, за који важи 

\begin{center}
	$ N = \sum_{i=0}^{m} s_{i}p_{i}, 0 \le s_{i} \le a_{i+1}, s_{i+1} = a_{i+2} => s_{i}=0 , i \ge 0 $
\end{center}

Слично важи и за $ q $-систем

\begin{center}
	$ N = \sum_{i=0}^{n} t_{i}q_{i}, 0 \le t_{0} \le a_{1}, 0 \le t_{i} \le a_{i+1}, t_{i} = a_{i+1} => t_{i-1}=0 , i \ge 1 $
\end{center}

Где су $ p_{i} $ и $ q_{i} $ $ i $-ти елементи горе дефинисаних низова $ p $ и $ q $, приказ првих неколико бројева записаних у $ p $ и $ q $ систему, за $ a_{i} = 2 , i \ge 1 $ дат је у табели \ref{tab:p_q_sistem}.

\begin{table}[h!]
	\caption{Приказ првих неколико бројева записаних у $ p $ и $ q $ систему, за $ a_{i} = 2 , i \ge 1 $}
	\label{tab:p_q_sistem}
	\begin{center}
		\begin{tabular}{ | c | c | c | c | c  c | c | c | c | c | c |}
			\hline
			{$ \mathbf{q_{3}} $} &  {$ \mathbf{q_{2}} $} &  {$ \mathbf{q_{1}} $} &  {$ \mathbf{q_{0}} $} & & &  {$ \mathbf{p_{3}} $} &  {$ \mathbf{p_{2}} $} &  {$ \mathbf{p_{1}} $} &  {$ \mathbf{p_{0}} $} &\\
			12 & 5 & 2 & 1 & & & 17 & 7 & 3 & 1 &  {$ \mathbf{n} $}\\
			\hline
			&  &  & 1 & & &  &  &  & 1 & {$ \mathbf{1} $}\\
			&  & 1 & 0 & & &  &  &  & 2 & {$ \mathbf{2} $}\\
			&  & 1 & 1 & & &  &  & 1 & 0 &  {$ \mathbf{3} $}\\
			&  & 2 & 0 & & &  &  & 1 & 1 &  {$ \mathbf{4} $}\\
			& 1 & 0 & 0 & & &  &  & 1 & 2 &  {$ \mathbf{5} $}\\
			& 1 & 0 & 1 & & &  &  & 2 & 0 &  {$ \mathbf{6} $}\\
			& 1 & 1 & 0 & & &  & 1 & 0 & 0 &  {$ \mathbf{7} $}\\
			& 1 & 1 & 1 & & &  & 1 & 0 & 1 &  {$ \mathbf{8} $}\\
			& 1 & 2 & 0 & & &  & 1 & 0 & 2 &  {$ \mathbf{9} $}\\
			& 2 & 0 & 0 & & &  & 1 & 1 & 0 &  {$ \mathbf{10} $}\\
			& 2 & 0 & 1 & & &  & 1 & 1 & 1 &  {$ \mathbf{11} $}\\
			 1 & 0 & 0 & 0 & & & & 1 & 1 & 2 &  {$ \mathbf{12} $}\\
			 1 & 0 & 0 & 1 & & & & 1 & 2 & 0 &  {$ \mathbf{13} $}\\
			 1 & 0 & 1 & 0 & & & & 2 & 0 & 0 &  {$ \mathbf{14} $}\\
			 1 & 0 & 1 & 1 & & & & 2 & 0 & 1 &  {$ \mathbf{15} $}\\
			 1 & 0 & 2 & 0 & & & & 2 & 0 & 2 &  {$ \mathbf{16} $}\\
			 1 & 1 & 0 & 0 & & & 1 & 0 & 0 & 0 &  {$ \mathbf{17} $}\\
			\hline 
		\end{tabular}
	\end{center}
\end{table}

Дефинисаћемо још \textit{репрезентацију $ R $} као $ (m+1) $-торка 
\begin{center}
	$ R = (d_{m}, d_{m-1}, ... , d_{1}, d_{0}), 0 \le d_{i} \le a_{i+1}, d_{i+1} = a_{i+2} => d_{i} = 0, i \ge 0 $.
\end{center}

Уколико у $ R $ померимо сваку цифру $ d_{i} $ у лево за једно место добијамо $ R' = (d_{m}, d_{m-1}, ... , d_{1}, d_{0}, 0) $, а уколико је $ R $ репрезентација са $ d_{0} = 0 $ онда када сваку цифру $ d_{i} $ померимо за једно место у десно добијамо $ R'' = (d_{m}, d_{m-1}, ... , d_{1}) $ 

$ I_{p} = \sum_{i=0}^{m} d_{i}p_{i} $ je \textit{$ p $-интерпретација} репрезентације $ R $.

$ I_{q} = \sum_{i=0}^{m} d_{i}q_{i} $ je \textit{$ q $-интерпретација} репрезентације $ R $.

Може се приказати и веза између рецимо $ p $-интерпретације и $ q $-репрезентације за позитиван број $ k $
\begin{center}
	$ I_{p}(R_{q}(k)) = I_{p}(d_{m}, d_{m-1}, ..., d_{1}) = n $
\end{center} 

На пример број $ R_{q}(12) = 1000 $, а $ I_{p}(1000) = 17 $, ово је приказано у табели \ref{tab:p_q_sistem}.

У случају када смо на позицији $ (x, y), 0 < x \le y $, прво је потребно ирачунати $ R_{p}(x) $ и проверити да ли се завршава са парним или непарним бројем нула.

Уколико се завршава са непарним бројем нула онда је $ x = B_{n} $, тако да је победнички потез $ (x, y) \rightarrow (I_{p}(R''_{p}(x)), x) $

Уколико се завршава са парним бројем нула онда је $ x = A_{n} $, ако је $ y > I_{p}(R'_{p}(x) $ победнички потез је $ (x, y) \rightarrow (x, I_{p}(R'_{p}(x)) $, иначе уколико је $ y < I_{p}(R'_{p}(x) $ рачунамо $ d = y - x, m = \lfloor \frac{d}{a} \rfloor $. Тако да уколико се сад $ R_{q}(m) $ завршава са парним бројем нула онда је $ A_{m} = I_{p}(R_{q}(m)) $, иначе уколико се завршава непарним бројем нула $ A_{m} = I_{p}(R_{q}(m)) - 1 $. У оба случаја победнички потез је $ (x, y) \rightarrow (A_{m}, A_{m} + ma) $
%Мало приче о верижним разломцима и да је 1 ово фибоначијев низ додати можда...

\section{Имплементација и евалуација}
\label{implementacija_evaluacija}

За сваку стратегију извршено је мерење конструкције П табеле, резултати извршавања у милисеундама зависно од $ n $, при фиксном $ a = 2 $ су приказани у табели \ref{tab:calculate_time_n}. 
За мерење је коришћена хроно библиотека (енг.{~\em chrono library}) \cite{chrono_library}. Сва мерења су извршена на раучунару са следећом конфигурацијом:
\begin{flushleft}
	CPU: Intel(R) Core(TM) i7-4510U CPU @ 2.00GHz\\
	RAM: Kingston 8GB 1600MHz DDR3\\
	OS: Debian GNU/Linux 9 (stretch)\\
	Compiler: gcc 6.3.0\\
\end{flushleft}

У табели \ref{tab:calculate_piles} приказане су величине парова жетона П табеле све до $ 10^{31} $, као и одговарајуће $ n $.

\subsection{Рекурзивна стратегија}

За раучунање П табеле рекурзивном стратегијом прво је потребно да израчунамо $ A_{i} $, тачније потребно је наћи најмањи позитиван број који до сада није у табели - $ mex $. За тражење је коришћен помоћни низ димензије $ 2*n $, иницијализиван нулама. Тражење $ mex $-а  своди се на проналажење индекса прве нуле, с обзиром да за елементе $ A $ важи $ a <= 2*n $ сложеност у најгорем случају је $ O(n) $. Чиме је укупна временска сложеност конструкције П табеле $ O(n^2) $.\\

\lstinputlisting[language=C++, linerange={7-31}, caption=Рекурзивна стратегија рачунање П табеле]{./src/recursive.cpp}

\leavevmode\\
Графички приказ зависности $ n $ и времена у милисекундама дат је на \ref{fig:recursive}, за $ а = 2 $.

\begin{figure}[H]
	\label{fig:recursive}
	\centering
	\includegraphics[width=\textwidth]{recursive.png}
\end{figure}

\subsection{Алгебарска стратегија}

За рзлику од рекурзиивне стратегије која користи имплицитну рекурзију, алгебарска стратегија користи експлицитну рекурзију, рачунајући $ alpha $ и $ beta $. Чиме је укупна временска сложеност конструкције П табеле $ O(n) $.\\

\lstinputlisting[language=C++, linerange={10-24}, caption=Алгебарска стратегија рачунање П табеле]{./src/algebraic.cpp}

\leavevmode\\
Графички приказ зависности $ n $ и времена у милисекундама дат је на \ref{fig:algebraic}, за $ а = 2 $.

\begin{figure}[H]
	\label{fig:algebraic}
	\centering
	\includegraphics[width=\textwidth]{algebraic.png}
\end{figure}

\subsection{Аритметичка стратегија}

За раучунање П табеле аритметичком стратегијом прво је потребно конструисати једноставан коначан верижни разломак, што захтева $ O(n) $ времена. 

Потом дефинишемо низове $ p $ и $ q $, њихова димензија је највише $ log(n) $, стога је врeме потребно да дефинишемо ове низове $ O(log(n)) $.

Преостаје још само да $ n $ бројева представимо у $ p $ и $ q $ систему, за њихово представљање у свакој итерацији имамо бинарну претрагу низова $ p $ и $ q $ којом се одређује са колико цифара треба представити број $ i $, што је у најгорем случају једнако величини низова $ p $ и $ q $, тачније $ log(n) $. Тако да је сложеност бинарне претраге $ O(log(log(n))) $. Репрезентација броја $ k $ у $ p $ или $ q $ систему се добија тако што рачунамо количник и остатак дељења броја $ k $ са одгварајућом вредности низа $ p $ или $ q $. Уколико имамо остатак потребно је и њега представити у $ p $ или $ q $ систему, његова $ p $ или $ q $ репрезентација је позната тако да је потребно само да је прекопирамо на крај текуће $ p $ или $ q $ репрезентације броја $ k $, не мењајући притом унапред дефинисан број цифара. Сложеност операције копирања једнака је броју елемената који се копира, што је у најгорем случају $ log(k) - 1 $ цифара. Како имамо $ n $ итерација укупна сложеност представљања првих $ n $ бројева у $ p $ и $ q $ систему захтева $ O(n(log(log(n)) + log(n) - 1)) $ времена.

Чиме је укупна временска сложеност конструкције П табеле $ O(nlog(n)) $

\lstinputlisting[language=C++, linerange={14-94}, caption=Аритметичка стратегија рачунање П табеле]{./src/arithmetic.cpp}

\leavevmode\\
Графички приказ зависности $ n $ и времена у милисекундама дат је на \ref{fig:arithmetic}, за $ а = 2 $.

\begin{figure}[H]
	\label{fig:arithmetic}
	\centering
	\includegraphics[width=\textwidth]{arithmetic.png}
\end{figure} 

\subsection{Сумиран приказ времена извршаваљњ свих стратегија}

Из претходне анализе се може закључити да је алгебарска стратегија најефикаснија, што се може видети и на обједињеним графицима \ref{fig:all} \ref{fig:algebraicVSarithmetic}.

\begin{figure}[H]
	\label{fig:all}
	\centering
	\includegraphics[width=\textwidth]{all.png}
\end{figure}

\begin{figure}[H]
	\label{fig:algebraicVSarithmetic}
	\centering
	\includegraphics[width=\textwidth]{algebraicVSarithmetic.png}
\end{figure}

\appendix
\section{Додатак резултатима}

\csvautolongtable[
table head=\caption{Времена извршавања конструкције П табеле}\label{tab:calculate_time_n}\\\hline
\csvlinetotablerow\\\hline
\endfirsthead\hline
\csvlinetotablerow\\\hline
\endhead\hline
\endfoot,
respect all
]{./src/statistics/csv/calculate_time_n.csv}

\csvautolongtable[
table head=\caption{Парови жетона П табеле}\label{tab:calculate_piles}\\\hline
\csvlinetotablerow\\\hline
\endfirsthead\hline
\csvlinetotablerow\\\hline
\endhead\hline
\endfoot,
respect all
]{./src/statistics/csv/calculate_piles.csv}

\addcontentsline{toc}{section}{Литература}
\appendix
\bibliography{literatura} 
\bibliographystyle{plain}


\end{document}
